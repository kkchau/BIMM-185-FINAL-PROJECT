\documentclass[12pt]{article}
\usepackage[utf8]{inputenc}
\usepackage{listings}
\usepackage{hyperref}
\usepackage{graphicx}


\lstdefinestyle{output}{
    breaklines=true,
    showspaces=false,
    basicstyle=\footnotesize,
    breakatwhitespace=false
}


\begin{document}

\lstset{style=output}
\setlength\parindent{0pt}

% title page
\begin{titlepage}
    \begin{center}
        \vspace*{1cm}

        \Huge
        \textbf{Predicting Pathogenicity using Aberrant Splicing Predictions of Single Nucleotide Variants in Autism Spectrum Disorder}\\


        \vspace{0.5cm}
        \Large
        \textbf{Bioinformatics Laboratory, BIMM 185}

        \vspace{1.5cm}
        \textbf{Kevin Chau}

        \vfill

        \Large
        University of California, San Diego\\
        16 June 2017
    \end{center}
\end{titlepage}

\thispagestyle{plain}
\begin{center}
    \textbf{Abstract}
\end{center}
Distributions of splicing likelihood scores for single nucleotide variants 
(SNVs) were analyzed with the hopes that these scores could be used to 
predict their pathogenicity, defined as a pathogenicity score multiplied by a 
risk score for that gene. SNVs implicated in autism as well as those found in
control phenotypes were gathered and scored for their likelihood to cause 
cassette exon skipping and their pathogenicity. Gene coexpression networks
were constructed for affected genes per eight developmental periods using
publicly available RNA-seq data. Genes were scored for risk, per period, 
based on relative numbers of coexpression partners. A poserior probability 
function was calculated as the predictive model relating splice score to
disease phenotype likelihood and was benchmarked for predicitve power.

\pagebreak

% begin report contents
\section{Introduction}
Prediction of pathogenicity of genetic variants is paramount to current 
understanding of diseases with heavy genetic influence and, by extension,
development of treatments. In addition, it has been shown that aberrant 
splicing can have major contributions to expression of disease phenotypes, 
likely due to a propensity to disrupt biological pathways through alteration 
of gene products. Therefore, one might deduce that variants 
predicted to cause alternative splicing in genes central to biological 
processes are likely to result in the expression of the pathogenic phenotype. 
With numerous tools currently available to the public for academic use, a 
predictive model may be constructed with the hopes that the likelihood that a 
genetic variant causes aberrant splicing can be used to predict the contribution 
of that variant to development of a given disease. This knowledge could provide 
new insight into therepeutic targets at the gene or even isoform level.

\section{Data and Processing}
A list of variants classified by expressed phenotype was downloaded from 
DenovoDB, a database of nucleotide variants. Only variants tagged with the
autism phenotype and control phenotype were downloaded. Overall, the 
taken from DenovoDB included the phenotype, the nucleotide variant, chromosome,
affected gene, and starting position of the variant. This table was uploaded
to a locally hosted MySQL database for simplified querying. Since only SNVs 
were analyzed, the dataset was filtered for only mutations that substitute
one nucleotide for another. The SPIDEX splicing scoring tool was downloaded 
in order to score the nucleotide variants for their likelihood to cause 
cassette exon skipping.

\subsection{Splicing Scores with SPIDEX}
The SPIDEX tool was used to score each SNV for the likelihood that
it causes cassette exon skipping. The scoring metric used was delta PSI (dpsi),
the change in percent inclusion rate; negative values correspond to decreased
inclusion (exon skipping) and positive values correspond to increased inclusion 
(exon retention). The criteria for scoring was that the SNV must have occurred
within 300 nucleotides from a splice junction; that is, within 300 
nucleotides from an exon-to-intron or intron-to-exon transition position. Since
not all variants fell within that distance, only scorable variants were 
retained for further analysis.
\\[\baselineskip]
The SPIDEX tool is downloaded as a tab-indexed gzip-compressed file. The contents of the
file was queried using the $tabix$ utility provided by the htslib C library 
(formerly packaged with samtools). Example query and output follows:

\lstinputlisting{imports/example_spidex.txt}

with fields as Chromosome, Position, Reference Allele, Mutant Allele, dPSI, 
and Gene. 

\subsection{Pathogenicity Scores with UMD Predictor}
Pathogenicity of the given variants were scored with the web tool UMD 
Predictor. This online tool takes in a list of formatted variants and positions
and returns a score for pathogenicity of the variant for each transcript 
affected. These scores ranged from 0 to 100, with 0 being non-pathogenic and
100 being confirmed pathogenicity. These scores were converted to percentages
in order to more closely conform with the splicing scores.

\subsection{Gene Networks and Risk Analysis}
Gene coexpression networks were constructed in order to take into account
gene risk factors, with the reasoning that genes with many coexpression 
partners are more likely to play an important role in their respective 
biological pathways; disruption in their splicing would thus lead to 
pathogenicity. These coexpression networks were created using RNA-seq 
expression data from the BrainSpan database. A single network was developed for
each of eight developmental periods, since expression values are likely to
vary depending on the given stage of life. These developmental periods are as
follows: 8 weeks postconception (PCW) to 12PCW, 13PCW to 18PCW, 19PCW to 23PCW,
24PCW to 37PCW, 0 months after birth (M) to 11M, 1 year (Y) to 11Y, 12Y to 19Y,
and 21Y$^{+}$. Pairwise comparisons were performed using the Pearson 
correlation metric and all gene-gene coexpression pairs with Pearson
correlation coeffections of $\geq 0.7$ were retained. The number of 
coexpression partners for each gene was calculated and divided by the maximum
of the set, yielding the risk score for that gene. These scores were appended
to all variants that affect that gene; the relevant variant information
along with the calculated scores were all uploaded to a local MySQL database.

\section{Data Analysis}
The final dataset to analyze was compiled into a MySQL database with the 
following layout:

\lstinputlisting[language=SQL]{imports/sql_example.txt}
\pagebreak

From this data, distributions could be plotted. First, a kernel density 
estimate was performed for the frequencies of splice scores for both the 
autism phenotype and control phenotype.

\begin{figure}[ht]
\centering
    \includegraphics[width=\textwidth,height=\textheight,keepaspectratio]{../analysis/splice_distributions.png}
    \caption{Kernel density estimates of autism SNV splice scores and control SNV splice scores}
\end{figure}

As shown, there seems to be a clear distinction between the distributions of 
splice scores between the autism phenotype and control phenotype.

\pagebreak

Scatter plots between splicing scores and pathogenicity scores
may also be drawn in order to illustrate any correlation between the two over
each developmental period and in each condition.

\begin{figure}[ht]
\centering
    \includegraphics[width=\textwidth,height=\textheight,keepaspectratio]{../analysis/pos_corr.png}
    \caption{Splicing probabilities (x-axis) plotted against pathogenicity probabilities (y-axis) for each developmental period in SNVs implicated in autism}
\end{figure}

\begin{figure}[ht]
\centering
    \includegraphics[width=\textwidth,height=\textheight,keepaspectratio]{../analysis/neg_corr.png}
    \caption{Splicing probabilities (x-axis) plotted against pathogenicity probabilities (y-axis) for each developmental period in control SNVs}
\end{figure}

\pagebreak

Clearly, since there is no correlation between the UMD Predictor predicted
pathogenicity scores, transformed by gene risk factors, and splice scores,
this data cannot be used as a prior probability. Therefore, a posterior 
probability function is calculated using prior probability of 0.5; that is,
an unweighted posterior probability function is created. Additionally, only a
random sample of half of the scores was used to create the posterior probability
function so as to better gauge the predictive power of the model through
benchmarking tests utilizing the various attributes of confusion matrix values
over a range of thresholding powers.

\begin{figure}[ht]
\centering
    \includegraphics[width=\textwidth,height=\textheight,keepaspectratio]{../analysis/posterior.png}
    \caption{Posterior probability function}
\end{figure}

\pagebreak

Benchmarking of the model was performed in order to gauge its predictive power.
Sensitivity and specificity plots were drawn to illustrate the performance
of the inference model.

\pagebreak

\begin{figure}[ht]
\centering
    \includegraphics[width=\textwidth,height=\textheight,keepaspectratio]{../analysis/svs.png}
    \caption{Sensitivity and specificity curves.}
\end{figure}

In addition, an accuracy curve was plotted to illustrate the relationship
between the true predictions and false predictions

\begin{figure}[ht]
\centering
    \includegraphics[width=\textwidth,height=\textheight,keepaspectratio]{../analysis/acc.png}
    \caption{Accuracy curve}
\end{figure}

\pagebreak

\section{Discussion}
Given the data as well as the benchmarking curves, the inference model proves
to be remarkably accurate. However, one should note that the model proposed in
this study is based on very specific criteria; only single nucleotide variants
within three hundred nucleotides of a splice junction in brain-expressed genes
that were able to be scored by the SPIDEX tool and were either implicated in 
autism spectrum disorder or were not associated with any other disease 
phenotype expression were considered out of the 
entire dataset. Additionally, the posterior function was calculated without
any prior probabilities; the likelihood that a single nucleotide variant could
be implicated in ASD versus the control phenotype were assumed to be the same.
Appropriation of this probabilistic model to other datasets should take these
notes into account.

\section{Supplementary Materials}
Relevant code:\\
\href{https://github.com/kkchau/BIMM-185-FINAL-PROJECT/}{https://github.com/kkchau/BIMM-185-FINAL-PROJECT/}

\end{document}
