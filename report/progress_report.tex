\documentclass[12pt]{article}
\usepackage[utf8]{inputenc}
\usepackage{listings}
\usepackage{hyperref}
\usepackage{graphicx}
\graphicspath{ {../analysis} }


\lstdefinestyle{output}{
    breaklines=true,
    showspaces=false,
    basicstyle=\footnotesize,
    breakatwhitespace=false
}


\begin{document}

\lstset{style=output}
\setlength\parindent{0pt}

% title page
\begin{titlepage}
    \begin{center}
        \vspace*{1cm}

        \Huge
        \textbf{Predicting Pathogenicity using Aberrant Splicing Predictions of Single Nucleotide Variants in Autism Spectrum Disorder}\\


        \vspace{0.5cm}
        \Large
        \textbf{Bioinformatics Laboratory, BIMM 185}

        \vspace{1.5cm}
        \textbf{Kevin Chau}

        \vfill

        \Large
        University of California, San Diego\\
        16 June 2017
    \end{center}
\end{titlepage}

% begin report contents
\section{Introduction}
The major question I wanted to answer was: "Can the likelihood that a single 
nucleotide variant causes cassette exon skipping be used as a predictor
for that nucleotide's potential for pathogenticity?" I hypothesized that 
there is a correlation between an SNV's propensity to cause alternative 
splicing and that variant's pathogenicity probability. The evidence to either 
support or reject my hypothesis was the result of processing scores of 
variants known to be involved in autism spectrum disorder and control variants.
This processing included scoring each variant for the predicted change in
splicing, as well as scoring each variant for a pathogenicity score.
Gene-gene coexpression networks were also constructed to calculate risk scores
for each gene affected by an SNV. Much of the data processing was done with
Python scripts; data management was performed using a local MySQL database.

\section{Data and Processing}
A list of variants classified by expressed phenotype was downloaded from 
DenovoDB, a database of nucleotide variants. Only variants tagged with the
autism phenotype and control phenotype were downloaded. Overall, the 
taken from DenovoDB included the phenotype, the nucleotide variant, chromosome,
affected gene, and starting position of the variant. This table was uploaded
to a locally hosted MySQL database for simplified querying. Since only SNVs 
were analyzed, the dataset was filtered for only mutations that substitute
one nucleotide for another. The SPIDEX splicing scoring tool was downloaded 
in order to score the nucleotide variants for their likelihood to cause 
cassette exon skipping.

\subsection{Splicing Scores with SPIDEX}
The downloaded SPIDEX tool was used to score each SNV for the likelihood that
it causes cassette exon skipping. The scoring metric used was delta PSI (dpsi),
the change in percent inclusion rate; negative values correspond to decreased
inclusion (exon skipping) and positive values correspond to increased inclusion 
(exon retention). The criteria for scoring was that the SNV must have occurred
within 300 nucleotides from a splice junction; that is, within 300 
nucleotides from an exon-to-intron or intron-to-exon transition position. Since
not all variants fell within that distance, only scorable variants were 
retained for further analysis.
\\[\baselineskip]
The SPIDEX tool is downloaded as a tab-indexed tar file. The contents of the
file was queried using the $tabix$ utility provided by the htslib C library 
(formerly packaged with samtools). Example query and output follows:

\lstinputlisting{imports/example_spidex.txt}

with fields as Chromosome, Position, Reference Allele, Mutant Allele, dPSI, 
and Gene.

\subsection{Pathogenicity Scores with UMD Predictor}
Pathogenicity of the given variants were scored with the web tool UMD 
Predictor. This online tool takes in a list of formatted variants and positions
and returns a score for pathogenicity of the variant for each transcript 
affected. These scores ranged from 0 to 100, with 0 being non-pathogenic and
100 being confirmed pathogenicity. These scores were converted to percentages
in order to more closely conform with the splicing scores.

\subsection{Gene Networks and Risk Analysis}
Gene coexpression networks were constructed in order to take into account
gene risk factors, with the reasoning that genes with many coexpression 
partners are more likely to play an important role in their respective 
biological pathways; disruption in their splicing would thus lead to 
pathogenicity. These coexpression networks were created using RNA-seq 
expression data from the BrainSpan database. A single network was developed for
each of eight developmental periods, since expression values are likely to
vary depending on the given stage of life. These developmental periods are as
follows: 8 weeks postconception (PCW) to 12PCW, 13PCW to 18PCW, 19PCW to 23PCW,
24PCW to 37PCW, 0 months after birth (M) to 11M, 1 year (Y) to 11Y, 12Y to 19Y,
and 21Y$^{+}$. Pairwise comparisons were performed using the Pearson 
correlation metric and all gene-gene coexpression pairs with Pearson
correlation coeffections of $\geq 0.7$ were retained. The number of 
coexpression partners for each gene was calculated and divided by the maximum
of the set, yielding the risk score for that gene. These scores were appended
to all variants that affect that gene; the relevant variant information
along with the calculated scores were all uploaded to a local MySQL database.

\section{Data Analysis}
The final dataset to analyze was compiled into a MySQL database with the 
following layout:

\lstinputlisting[language=SQL]{imports/sql_example.txt}

Actual analysis is yet to be performed and will be the subject of this week's 
work.
\\[\baselineskip]
My GitHub with relevant code can be found here: 
\href{https://github.com/kkchau/bimm185/}{https://github.com/kkchau/bimm185/}
\end{document}
